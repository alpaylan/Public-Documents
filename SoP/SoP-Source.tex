\documentclass{letter}
\usepackage{hyperref}
\usepackage[margin = 0.8in]{geometry}
\begin{document}
\begin{letter}{}
\opening{\textbf{Research Summary: Alperen Keleş}}


The key to a good story is a grand opening. Hence, I want to start from the inception. Although I did not know what Ph.D. stands for at the time, I knew I wanted to do research since high school.  It all began when I started studying for National Informatics Olympiads on Computer Science. The problems, the thinking, and the challenges intrigued me, yet I was not too fond of the field's practical side, which led me to change my course. I used my background in programming and algorithms to participate in National High School Research Projects. I did two projects, "Using Expert Systems and Fuzzy Logic for Disease Prediction" and "Smart Personal Assistant ALLISON". The projects lacked technical depth and were illusions of a high school student at best, but they are the first steps invoking my interest in research and leading me to write this statement of purpose today. 

As I began undergraduate studies, I searched for opportunities to work on research projects. I asked a professor of mine. He responded by saying this was too early for me. Even though I see this as a setback in my technical development, I used it as an opportunity for social and intellectual improvement. I started competitive debating; later, I joined the local IEEE Society after taking part in their hackathon. For almost two years, I have worked in student organizations. We have organized hackathons, designed engineering camps for high school students to learn about STEM fields, created a science blog written entirely by students. Aside from organizations, I also had weekly meetings that I designed the original content. I made role plays, invented a language, Kelesce(Language of Keles in Turkish), and formulated murder solving puzzles. I have spent two years learning how to communicate effectively, teach, persuade, and influence people.  

Besides these activities, I also worked on my technical development. I have been polishing my data structures and algorithms skills for competitive programming. I participated in ACM ICPC in Ukraine, had the opportunity to open my eyes to a view of the world, leaving Turkey for the first time. 

But when I finished my third semester at METU, I decided that it's time to start participating in research. I have communicated with Asst. Prof. Hande Alemdar. She was kind enough to invite me to a research project on Enhancing Capabilities of Network Intrusion Detection Systems Using Machine Learning. After studying for a semester, I wrote a research paper and presented my poster on Using Random Forest on CICIDS2017 for NIDS. During this study, I learned how to delve into a new field, find my way inside a vast space of information and literature, read a paper, and critique various aspects such as the originality and the experiments. I also grasped the rigorous elements of writing a research paper, framing our contributions, and performing reproducible experiments. We later integrated this work into another study in our department with Asst. Prof. Pelin Angın, DroPPPP: A P4 Approach to Mitigating DoS Attacks in SDN. We took our rule based findings from decision trees about different attack types, mainly DOS/DDOS, and worked on using them for the firewall developed in DroPPPP: A P4 Approach to Mitigating DoS Attacks in SDN. 

I have met with the intersection of formal verification in my first internship in Germany at Emproof. I learned about computer-aided formal verification,  model checking and how formal verification is used in embedded security. I worked for three months on generating an intermediate representation and verification of obfuscated assembly programs. I also had the opportunity to live alone in a foreign country for the first time in my life. My ideals, objectives, and vision have all changed a lot during this duration. I have developed a particular interest in verification, programming languages, research and an aim to develop myself and discover my potential. After returning to Turkey, I have turned this interest into a research project where I designed PVL(Protocol Verification Language) with a group of four. 

PVL has been a gold mine since its creation. It was solely a student project without a supervisor, which led us to learn by making mistakes. Each one of our faults has allowed us to understand more about the project and the field. We learned design principles and architectural planning. From the build system to the verification engine, we have designed and implemented a whole new programming language, every step was more informative than the former. Aside from teaching aspects, I became a Falling Walls finalist by presenting the project with the headline Breaking The Wall of Software Bugs. As a Falling Walls Finalist, I had the opportunity to meet with research scientists worldwide; I was lucky to broaden my vision by seeing other projects and speaking with the finalists. Also, I had my first peer-reviewed research accepted as the first author. Today, we continue the project with a supervisor, Asst. Prof. Ebru Aydın Göl, as our final project. Also, we are currently preparing a research paper for submission to CAV 2021. 

After my first internship in the industry, I have shown an effort for summer practice in research. I did my second internship with Tudor Dumitraş at Maryland Cybersecurity Center. This internship centered around the Application of Program Synthesis Techniques in Automatic Exploit Generation. I have worked closely with Tudor and one of his Ph.D. students, Erin Avllazagaj. I started my internship by working on applying automated reasoning on heap allocator behaviour using inductive program synthesis techniques, later we focused the objective of the project into different domains in automatic exploit primitive detection. Throughout the internship, I got the opportunity to meet with other PhD students and professors from different universities, made several presentations on my work and area. I was really lucky to have a simulation of a Ph.D. life for three months, and I loved it. 

After five years of studying competitive programming, three years of studying computer engineering, two internships in industry and academy, four different projects in different domains, I had enough experience to decide my future. I knew that I was meant to be a researcher for the rest of my life, and I knew that I wanted to do my Ph.D. centered around formal verification and analysis. One of the most important goals of my life is that I love what I do, and I love a vibrant and multi-focused life. I love combining my experience and knowledge from different fields and applying formalism to those areas. As a researcher working on formal verification and analysis, I can apply work on the intersection of various domains as I have experienced with PVL; with the formalism, dynamism, and multi-disciplinarity of the area, I believe it is the best fit among all the fields I previously discovered. 

I have been working on verification and logic for quite some time. I have taken part in various projects, taken various courses in both undergraduate and graduate level, I am highly interested in these topics. An important thing to point out about my experiences is that they have been on a very large spectrum of verification and analysis. I have worked on verification and analysis of low level ARM assembly, then moved on to PVL where we worked on verification of functions and programs, learned about verification using LTL and CTL, took courses on AI and Logic. Throughout this journey, I never felt tired nor bored, but always enthusiast about new topics and areas, I am confident that I want to my Ph.D. on verification. 

I can see a clear future for myself working on Gillian, a multi-language platform for compositional symbolic analysis developed by  Professor Philippa Gardner’s group working on `Verified Software’. Gillian unifies symbolic testing and verification based on Separation Logic (SL),  and can be instantiated to a wide range of programming languages (currently JavaScript and C).  Given my prior experience, I would like to work on the core of  Gillian especially considering  that I love working on the theory/practice intersection.  I have read a few papers on SL prior to my interviews with Professor Gardner. I think SL  is a topic with a wide open future and interesting research questions to work with regarding limitations, optimizations and extensions of SL. Even with the small amount of information I acquired from my reading and talking to Professor Gardner, I was excited by the idea of SL and I would love to learn more. Also, due to my previous experiences in C (7 years) and Rust (designing and implementing the compiler and the verification engine for PVL), I could also be a good fit for working on  the Gillian instantiation in C and  the planned instantiation in Rust. In my first internship, I  worked on a low-level intermediate language for verification based on REIL(Reverse Engineering Intermediate Language). This experience motivates me that I could also work on GIL, the intermediate language of Gillian. 


Department of Computing at Imperial College London possesses a wide range of strong academics working on verification and analysis, formal methods and software engineering. I believe that with the guidance of Professor Philippa Gardner and within the atmosphere of scientific community in the department, I can overcome the rigorous challenges in the path to becoming a veteran scientist, broaden my horizon and enlarge my potential. Professor Gardner's sincere attitute during our meetings combined with her experience and the topics she works on are more than enough to excite me in a significant amount. I believe while learning about Separation Logic and working with various parts of Gillian, I can thrive in a working environment with Professor Gardner and at Imperial College London.  

Even though the key is the opening, the closing is the unlocking of a great story. When stripped from all of my experiences and accomplishments, what is left behind is a raw love for learning, a heated mind of curiosity, and a non-ending desire to affect the lives of others in a better way. I believe that these are the qualities that make me a good researcher, and they are what makes me wake up in the morning, they are what makes me hopeful for tomorrow, and they are also what makes me write this letter. 

\end{letter}
\end{document}